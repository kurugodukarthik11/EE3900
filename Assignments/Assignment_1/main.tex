\documentclass[journal,12pt,twocolumn]{IEEEtran}
%
\usepackage{setspace}
\usepackage{gensymb}
\usepackage{xcolor}
\usepackage{caption}
%\usepackage{subcaption}
%\doublespacing
\singlespacing

%\usepackage{graphicx}
%\usepackage{amssymb}
%\usepackage{relsize}
\usepackage[cmex10]{amsmath}
\usepackage{mathtools}
%\usepackage{amsthm}rt
%\interdisplaylinepenalty=2500
%\savesymbol{iint}
%\usepackage{txfonts}
%\restoresymbol{TXF}{iint}
%\usepackage{wasysym}
\usepackage{hyperref}
\usepackage{amsthm}
\usepackage{mathrsfs}
\usepackage{txfonts}
\usepackage{stfloats}
\usepackage{cite}
\usepackage{cases}
\usepackage{subfig}
%\usepackage{xtab}
\usepackage{longtable}
\usepackage{multirow}
%\usepackage{algorithm}
%\usepackage{algpseudocode}
%\usepackage{enumerate}
\usepackage{enumitem}
\usepackage{mathtools}
\usepackage{polynom}
%\usepackage{iithtlc}
%\usepackage[framemethod=tikz]{mdframed}
\usepackage{listings}



%\usepackage{stmaryrd}


%\usepackage{wasysym}
%\newcounter{MYtempeqncnt}
\DeclareMathOperator*{\Res}{Res}
%\renewcommand{\baselinestretch}{2}
\renewcommand\thesection{\arabic{section}}
\renewcommand\thesubsection{\thesection.\arabic{subsection}}
\renewcommand\thesubsubsection{\thesubsection.\arabic{subsubsection}}

\renewcommand\thesectiondis{\arabic{section}}
\renewcommand\thesubsectiondis{\thesectiondis.\arabic{subsection}}
\renewcommand\thesubsubsectiondis{\thesubsectiondis.\arabic{subsubsection}}

%\renewcommand{\labelenumi}{\textbf{\theenumi}}
%\renewcommand{\theenumi}{P.\arabic{enumi}}

% correct bad hyphenation here
\hyphenation{op-tical net-works semi-conduc-tor}

\lstset{
language=Python,
frame=single, 
breaklines=true,
columns=fullflexible
}



\begin{document}
%

\theoremstyle{definition}
\newtheorem{theorem}{Theorem}[section]
\newtheorem{problem}{Problem}
\newtheorem{proposition}{Proposition}[section]
\newtheorem{lemma}{Lemma}[section]
\newtheorem{corollary}[theorem]{Corollary}
\newtheorem{example}{Example}[section]
\newtheorem{definition}{Definition}[section]
%\newtheorem{algorithm}{Algorithm}[section]
%\newtheorem{cor}{Corollary}
\newcommand{\BEQA}{\begin{eqnarray}}
\newcommand{\EEQA}{\end{eqnarray}}
\newcommand{\define}{\stackrel{\triangle}{=}}
\newcommand{\myvec}[1]{\ensuremath{\begin{pmatrix}#1\end{pmatrix}}}
\newcommand{\mydet}[1]{\ensuremath{\begin{vmatrix}#1\end{vmatrix}}}
\bibliographystyle{IEEEtran}
%\bibliographystyle{ieeetr}
\providecommand{\nCr}[2]{\,^{#1}C_{#2}} % nCr
\providecommand{\nPr}[2]{\,^{#1}P_{#2}} % nPr
\providecommand{\mbf}{\mathbf}
\providecommand{\pr}[1]{\ensuremath{\Pr\left(#1\right)}}
\providecommand{\qfunc}[1]{\ensuremath{Q\left(#1\right)}}
\providecommand{\sbrak}[1]{\ensuremath{{}\left[#1\right]}}
\providecommand{\lsbrak}[1]{\ensuremath{{}\left[#1\right.}}
\providecommand{\rsbrak}[1]{\ensuremath{{}\left.#1\right]}}
\providecommand{\brak}[1]{\ensuremath{\left(#1\right)}}
\providecommand{\lbrak}[1]{\ensuremath{\left(#1\right.}}
\providecommand{\rbrak}[1]{\ensuremath{\left.#1\right)}}
\providecommand{\cbrak}[1]{\ensuremath{\left\{#1\right\}}}
\providecommand{\lcbrak}[1]{\ensuremath{\left\{#1\right.}}
\providecommand{\rcbrak}[1]{\ensuremath{\left.#1\right\}}}
\theoremstyle{remark}
\newtheorem{rem}{Remark}
\newcommand{\sgn}{\mathop{\mathrm{sgn}}}
\providecommand{\abs}[1]{\left\vert#1\right\vert}
\providecommand{\res}[1]{\Res\displaylimits_{#1}} 
\providecommand{\norm}[1]{\lVert#1\rVert}
\providecommand{\mtx}[1]{\mathbf{#1}}
\providecommand{\mean}[1]{E\left[ #1 \right]}
\providecommand{\fourier}{\overset{\mathcal{F}}{ \rightleftharpoons}}
\providecommand{\ztrans}{\overset{\mathcal{Z}}{ \rightleftharpoons}}
%\providecommand{\hilbert}{\overset{\mathcal{H}}{ \rightleftharpoons}}
\providecommand{\system}{\overset{\mathcal{H}}{ \longleftrightarrow}}
	%\newcommand{\solution}[2]{\textbf{Solution:}{#1}}
\newcommand{\solution}{\noindent \textbf{Solution: }}
\providecommand{\dec}[2]{\ensuremath{\overset{#1}{\underset{#2}{\gtrless}}}}
\numberwithin{equation}{section}
%\numberwithin{equation}{subsection}
%\numberwithin{problem}{subsection}
%\numberwithin{definition}{subsection}
%\newcommand{\myvec}[1]{\ensuremath{\begin{pmatrix}#1\end{pmatrix}}}
\newcommand{\mybvec}[1]{\ensuremath{\begin{bmatrix}#1\end{bmatrix}}}
\newcommand{\w}[2]{\ensuremath{W_{#1}^{#2}}}
%\newcommand{\solution}{\noindent \textbf{Solution: }}
\providecommand{\dec}[2]{\ensuremath{\overset{#1}{\underset{#2}{\gtrless}}}}
\numberwithin{equation}{section}
%\numberwithin{equation}{subsection}
%\numberwithin{problem}{subsection}
%\numberwithin{definition}{subsection}
\let\vec\mathbf	
\makeatletter
\@addtoreset{figure}{problem}
\makeatother
\let\StandardTheFigure\thefigure
%\renewcommand{\thefigure}{\theproblem.\arabic{figure}}
\renewcommand{\thefigure}{\theproblem}
%\numberwithin{figure}{subsection}
\def\putbox#1#2#3{\makebox[0in][l]{\makebox[#1][l]{}\raisebox{\baselineskip}[0in][0in]{\raisebox{#2}[0in][0in]{#3}}}}
     \def\rightbox#1{\makebox[0in][r]{#1}}
     \def\centbox#1{\makebox[0in]{#1}}
     \def\topbox#1{\raisebox{-\baselineskip}[0in][0in]{#1}}
     \def\midbox#1{\raisebox{-0.5\baselineskip}[0in][0in]{#1}}
\vspace{3cm}

\title{ 
Digital Signal Processing
}

\author{ Karthik Kurugodu$^{*}$ 
}


% make the title area
\maketitle

%\newpage

\tableofcontents

%\renewcommand{\thefigure}{\thesection.\theenumi}
%\renewcommand{\thetable}{\thesection.\theenumi}

\renewcommand{\thefigure}{\theenumi}
\renewcommand{\thetable}{\theenumi}

%\renewcommand{\theequation}{\thesection}


\bigskip

\begin{abstract}
This manual provides a simple introduction to digital signal processing.
\end{abstract}
\section{Software Installation}
Run the following commands
\begin{lstlisting}
sudo apt-get update
sudo apt-get install libffi-dev libsndfile1 python3-scipy  python3-numpy python3-matplotlib 
sudo pip install cffi pysoundfile 
\end{lstlisting}
\section{Digital Filter}
\begin{enumerate}[label=\thesection.\arabic*
,ref=\thesection.\theenumi]
\item
\label{prob:input}
Download the sound file from  
\begin{lstlisting}
wget https://raw.githubusercontent.com/gadepall/
EE1310/master/filter/codes/Sound_Noise.wav
\end{lstlisting}
\item
\label{prob:spectrogram}
You will find a spectrogram at \href{https://academo.org/demos/spectrum-analyzer}{\url{https://academo.org/demos/spectrum-analyzer}}. 
Upload the sound file that you downloaded in Problem \ref{prob:input} in the spectrogram  and play.  Observe the spectrogram. What do you find?
\\
%
\solution There are a lot of yellow lines between 440 Hz to 5.1 KHz.  These represent the synthesizer key tones. Also, the key strokes
are audible along with background noise.
% By observing spectrogram, it clearly shows that tonal frequency is under 4kHz. And above 4kHz only noise is present.
\item
\label{prob:output}
Write the python code for removal of out of band noise and execute the code.
\\
\solution
\lstinputlisting{./codes/Cancel_noise.py}
\item
The output of the python script in Problem \ref{prob:output} is the audio file Sound\_With\_ReducedNoise.wav. Play the file in the spectrogram in Problem \ref{prob:spectrogram}. What do you observe?
\\
\solution The key strokes as well as background noise is subdued in the audio.  Also,  the signal is blank for frequencies above 5.1 kHz.
\end{enumerate}
\section{Difference Equation}
\begin{enumerate}[label=\thesection.\arabic*,ref=\thesection.\theenumi]
\item Let
\label{def:xn}
\begin{equation}
x(n) = \cbrak{\underset{\uparrow}{1},2,3,4,2,1}
\end{equation}
Sketch $x(n)$.
\item Let
\begin{multline}
\label{eq:iir_filter}
y(n) + \frac{1}{2}y(n-1) = x(n) + x(n-2), 
\\
 y(n) = 0, n < 0
\end{multline}
Sketch $y(n)$.
\\
\solution The following code yields Fig. \ref{fig:xnyn}.
\begin{lstlisting}
wget https://github.com/gadepall/EE1310/raw/master/filter/codes/xnyn.py
\end{lstlisting}
% \lstinputlisting{./codes/xnyn.py}
\begin{figure}[!ht]
\begin{center}
\includegraphics[width=\columnwidth]{./figs/xnyn}
\end{center}
\captionof{figure}{}
\label{fig:xnyn}	
\end{figure}
\item Repeat the above exercise using a C code.
\\
\solution The following code yields Fig. \ref{fig:xnyn}.
\begin{lstlisting}
wget https://github.com/kurugodukarthik11/EE3900/blob/main/Assignments/Assignment_1/codes/xny.cpp
\end{lstlisting}
\begin{figure}[!ht]
\begin{center}
\includegraphics[width=\columnwidth]{./figs/xny}
\end{center}
\captionof{figure}{}
\label{fig:xnyn}	
\end{figure}\bigskip\bigskip
\end{enumerate}

\section{$Z$-transform}
\begin{enumerate}[label=\thesection.\arabic*]
\item The $Z$-transform of $x(n)$ is defined as
%
\begin{equation}
\label{eq:z_trans}
X(z)={\mathcal {Z}}\{x(n)\}=\sum _{n=-\infty }^{\infty }x(n)z^{-n}
\end{equation}
%
Show that
\begin{equation}
\label{eq:shift1}
{\mathcal {Z}}\{x(n-1)\} = z^{-1}X(z)
\end{equation}
and find
\begin{equation}
	{\mathcal {Z}}\{x(n-k)\} 
\end{equation}
\solution From \eqref{eq:z_trans},
\begin{align}
{\mathcal {Z}}\{x(n-1)\} &=\sum _{n=-\infty }^{\infty }x(n-1)z^{-n}
\\
&=\sum _{n=-\infty }^{\infty }x(n)z^{-n-1} = z^{-1}\sum _{n=-\infty }^{\infty }x(n)z^{-n}
\end{align}
Similarly,
%
\begin{align}
{\mathcal {Z}}\{x(n-k)\} &=\sum _{n=-\infty }^{\infty }x(n-k)z^{-n}
\\
&=\sum _{n=-\infty }^{\infty }x(n)z^{-n-k} 
\\
&= z^{-k}\sum _{n=-\infty }^{\infty }x(n)z^{-n}
\end{align}
\begin{equation}
\label{eq:z_trans_shift}
	{\mathcal {Z}}\{x(n-k)\} = z^{-k}X(z)
\end{equation}
\item Obtain $X(z)$ for $x(n)$ defined in problem 
	\ref{def:xn}.
\\
\\
\solution
We know from \eqref{eq:z_trans}
\begin{equation}
\label{eq:e1}
X(z)={\mathcal {Z}}\{x(n)\}=\sum _{n=-\infty }^{\infty }x(n)z^{-n}
\end{equation}
Substitute values of $x(n)$ from \eqref{def:xn} in \eqref{eq:z_trans}
\begin{align}
    X(z)&={\mathcal {Z}}\{x(n)\}
    \\
    &=\sum _{n=-\infty }^{\infty }x(n)z^{-n}
    \\
    X(z)&=\sum _{n=0 }^{5 }x(n)z^{-n}
\end{align}
\item Find
%
\begin{equation}
H(z) = \frac{Y(z)}{X(z)}
\end{equation}
%
from  \eqref{eq:iir_filter} assuming that the $Z$-transform is a linear operation.
\\
\solution  Applying \eqref{eq:z_trans_shift} in \eqref{eq:iir_filter},
\begin{align}
Y(z) + \frac{1}{2}z^{-1}Y(z) &= X(z)+z^{-2}X(z)
\\
\implies \frac{Y(z)}{X(z)} &= \frac{1 + z^{-2}}{1 + \frac{1}{2}z^{-1}}
\label{eq:freq_resp}
\end{align}
%
\item Find the Z transform of 
\begin{equation}
\delta(n)
=
\begin{cases}
1 & n = 0
\\
0 & \text{otherwise}
\end{cases}
\end{equation}
and show that the $Z$-transform of
\begin{equation}
\label{eq:unit_step}
u(n)
=
\begin{cases}
1 & n \ge 0
\\
0 & \text{otherwise}
\end{cases}
\end{equation}
is
\begin{equation}
U(z) = \frac{1}{1-z^{-1}}, \quad \abs{z} > 1
\end{equation}
\solution
\begin{align}
{\mathcal {Z}}\{\delta(n)\} &=\sum _{n=-\infty }^{\infty }\delta(n)z^{-n}
\\
&= \delta(0)z^{0}
\\
&= 1
\end{align}
Thus
\begin{equation}
\delta(n) \ztrans 1
\end{equation}
and from \eqref{eq:unit_step},
\begin{align}
U(z) &= \sum _{n= 0}^{\infty}z^{-n}
\\
&=\frac{1}{1-z^{-1}}, \quad \abs{z} > 1
\end{align}
using the fomula for the sum of an infinite geometric progression.
%
\bigskip
\item Show that 
\begin{equation}
\label{eq:anun}
a^nu(n) \ztrans \frac{1}{1-az^{-1}} \quad \abs{z} > \abs{a}
\end{equation}
\solution
Infinite GP:
\begin{align}
\sum _{n= 0}^{\infty}r^n = \frac{1}{1-r}, \quad \abs{r} < 1
\end{align}
\begin{align}
U(z) &= \sum _{n= 0}^{\infty}a^nz^{-n}
\\
&=\frac{1}{1-az^{-1}}, \quad \abs{z} > \abs{a}
\end{align}
%
\item 
Let
\begin{equation}
    H\brak{e^{\j \omega}} = H\brak{z = e^{\j \omega}}.
\end{equation}
Plot $\abs{H\brak{e^{\j \omega}}}$.  Is it periodic? If so, find the period. $H(e^{\j \omega})$ is
known as the {\em Discret Time Fourier Transform} (DTFT) of $x(n)$.
\\
\solution
\begin{align}
	H\brak{e^{\j \omega}} &= \frac{1+e^{-2\j\omega}}{1+\frac{1}{2} e^{-\j\omega}} \\
	\abs{H\brak{e^{\j \omega}}} &= \frac{\abs{1+\cos2\omega - \j\sin2\omega}}{\abs{1+\frac{1}{2} \cos\omega - \frac{1}{2} \sin\omega}} \\
	&= \sqrt{\frac{(1+\cos2\omega)^2 + (\sin2\omega)^2}{(1 + \frac{1}{2} \cos\omega)^2 + (\frac{1}{2}\sin\omega)^2}} \\
	&= \sqrt{\frac{2+2\cos2\omega}{\frac{5}{4}+\cos\omega}} \\
	&= \sqrt{\frac{2(2\cos^2\omega)4}{5+4\cos\omega} } \\
	&= \frac{4\abs{\cos\omega}}{\sqrt{5+4\cos\omega}}
\end{align} 
Therefore the period of above equation is $2\pi$ since lcm of period of numerator($\pi$) and period of denominator($2\pi$) is $2\pi$.\\
\\The following code plots Fig\ref{fig:dtft}.
\begin{lstlisting}
wget https://github.com/kurugodukarthik11/EE3900/blob/main/Assignments/Assignment_1/codes/dtft.py
\end{lstlisting}
\begin{figure}[!ht]
\centering
\includegraphics[width=\columnwidth]{./figs/dtft}
\caption{$\abs{H\brak{e^{\j\omega}}}$}
\label{fig:dtft}
\end{figure} 
\item Express $h(n)$ in terms of $H\brak{e^{\j \omega}}$.
\solution We have,
\begin{align}
    H(e^{\text{\j}\omega}) &= \sum_{k = -\infty}^{\infty}h(k)e^{-\text{\j}\omega k}
\end{align}
However,
\begin{align}
    \int_{-\pi}^{\pi}e^{\text{\j}\omega(n - k)}d\omega =
    \begin{cases}
        2\pi & n = k \\
        0 & \textrm{otherwise}
    \end{cases}
\end{align}
and so,
\begin{align}
    &\frac{1}{2\pi}\int_{-\pi}^{\pi}H(e^{\text{\j}\omega})e^{\text{\j}\omega n}d\omega \\
    &= \frac{1}{2\pi}\sum_{k = -\infty}^{\infty}\int_{-\pi}^{\pi}h(k)e^{\text{\j}\omega(n - k)}d\omega \\
    &= \frac{1}{2\pi}2\pi h(n) = h(n)
\end{align}
which is known as the Inverse Discrete Fourier Transform. Thus,
\begin{align}
    h(n) &= \frac{1}{2\pi}\int_{-\pi}^{\pi}H(e^{\text{\j}\omega})e^{\text{\j}\omega n}d\omega \\
         &= \frac{1}{2\pi}\int_{-\pi}^{\pi}\frac{1 + e^{-2\text{\j}\omega}}{1 + \frac{1}{2}e^{-\text{\j}\omega}}e^{\text{\j}\omega n}d\omega
    \label{eq:idtft}
\end{align}  
\end{enumerate}

\section{Impulse Response}
\begin{enumerate}[label=\thesection.\arabic*]
	\item Using long division, find
\begin{align}
    h(n), \quad n < 5
\end{align}
for H(z) in 
\eqref{eq:freq_resp}.\\
\\
\solution For long division, substitute $x := z^{-1}$
\polylongdiv{1 + x^2}{1 + \frac{1}{2}x}
\\
Therefore,
\begin{align}
 H(z) &= -4 + 2z^{-1} + \frac{5}{1 + \frac{1}{2}z^{-1}} \\
   &= -4 + 2z^{-1} + 5\sum_{n = 0}^{\infty}\brak{-\frac{1}{2}}^nz^{-n} \\
   &= 1 - \frac{1}{2}z^{-1} + 5\sum_{n = 2}^{\infty}\brak{-\frac{1}{2}}^nz^{-n} \\
   &= \sum_{n = 0}^{\infty}\brak{-\frac{1}{2}}^nz^{-n} + 4\sum_{n = 2}^{\infty}\brak{-\frac{1}{2}}^nz^{-n} 
   \end{align}
   \begin{equation}
       = \sum_{n = -\infty}^{\infty}u(n)\brak{-\frac{1}{2}}^nz^{-n} + \sum_{n = -\infty}^{\infty}u(n - 2)\brak{-\frac{1}{2}}^{n - 2}z^{-n}
   \end{equation}   
Now, from \eqref{eq:z_trans}, we get
\begin{align}
 h(n) = \brak{-\frac{1}{2}}^{n}u(n) + \brak{-\frac{1}{2}}^{n-2}u(n-2)
\end{align}
\item \label{prob:impulse_resp}
Find an expression for $h(n)$ using $H(z)$, given that 
%in Problem \ref{eq:ztransab} and \eqref{eq:anun}, given that
\begin{equation}
\label{eq:impulse_resp}
h(n) \ztrans H(z)
\end{equation}
and there is a one to one relationship between $h(n)$ and $H(z)$. $h(n)$ is known as the {\em impulse response} of the
system defined by \eqref{eq:iir_filter}.
\\
\solution From \eqref{eq:freq_resp},
\begin{align}
H(z) &= \frac{1}{1 + \frac{1}{2}z^{-1}} + \frac{ z^{-2}}{1 + \frac{1}{2}z^{-1}}
\\
\implies h(n) &= \brak{-\frac{1}{2}}^{n}u(n) + \brak{-\frac{1}{2}}^{n-2}u(n-2)
\end{align}
ROC: (-$\infty$,-1/2) $\cup$ (-1/2,$\infty$)
\\
using \eqref{eq:anun} and \eqref{eq:z_trans_shift}.
\item Sketch $h(n)$. Is it bounded? Justify theoretically.
\\
\solution 
\\
We know 
\begin{equation}
 a)   \abs{\brak{ \frac{-1}{2}}^n} \leq 1
\end{equation}
\begin{equation}
 b)   \abs{u(n)} \leq 1
\end{equation}
\begin{equation}
 c)   \abs{\brak{ \frac{-1}{2}}^{n-2}} \leq 1
\end{equation}
\begin{equation}
 d)   \abs{u(n-2)} \leq 1
\end{equation}
Therefore 
\begin{equation}
    \abs{\brak{ \frac{-1}{2}}^n u(n)} \leq 1
\end{equation}
\begin{equation}
    \abs{\brak{ \frac{-1}{2}}^{n-2} u(n-2)} \leq 1
\end{equation}

Therefore
\begin{equation}
    % \abs{{\brak{ \brak{\frac{-1}{2}}^\brak{n-2}} u(n-2)} + \brak{ \frac{-1}{2}}^n u(n)} \leq 2
    \abs{\brak{ \frac{-1}{2}}^n u(n) + \brak{ \frac{-1}{2}}^{n-2} u(n-2)} \leq 2
\end{equation}

Hence Bounded
\\
The following code plots Fig. \ref{fig:hn}.
\begin{lstlisting}
wget https://github.com/kurugodukarthik11/EE3900/blob/main/Assignments/Assignment_1/codes/hn.py
\end{lstlisting}
\begin{figure}[!ht]
\centering
\includegraphics[width=\columnwidth]{./figs/hn.pdf}
\caption{$h(n)$ as the inverse of $H(z)$}
\label{fig:hn}
\end{figure}
%
\item Convergent? Justify using the ratio test.\\
\solution
Using the ratio test for convergence\\
        \begin{equation}
            \lim_{n \to \infty} \abs{\frac{h(n+1)}{h(n)}}
        \end{equation}
        \begin{equation}
            = \lim_{n \to \infty} \abs{\frac{\brak{-\frac12}^{n+1} u(n+1) + \brak{-\frac12}^{n-1} u(n-1) }{\brak{-\frac12}^{n} u(n) + \brak{-\frac12}^{n-2} u(n-2) }}
        \end{equation}
        \begin{equation}
            \leq \lim_{n \to \infty} \abs{\frac{\brak{-\frac12}^{n-1} \brak{\frac14 + 1}}{\brak{-\frac12}^{n-2} \brak{\frac14 + 1}}}
        \end{equation}
        \begin{equation}
            = \lim_{n \to \infty} \abs{-\frac12}
        \end{equation}
        \begin{equation}
            =\frac{1}{2} < 1
        \end{equation}  
\\
Hence by ratio test $h(n)$ is convergent\\
\item The system with $h(n)$ is defined to be stable if
\begin{equation}
\label{eq:stable}
\sum_{n=-\infty}^{\infty}h(n) < \infty
\end{equation}
Is the system defined by \eqref{eq:iir_filter} stable for the impulse response in \eqref{eq:impulse_resp}?
\\
\solution
\begin{align}
% H(z) &= \frac{1}{1 + \frac{1}{2}z^{-1}} + \frac{ z^{-2}}{1 + \frac{1}{2}z^{-1}}
\\
\sum_{n=-\infty}^{\infty}h(n) &= \sum_{n=-\infty}^{\infty}\brak{-\frac{1}{2}}^{n}u(n) + 
\sum_{n=-\infty}^{\infty}\brak{-\frac{1}{2}}^{n-2}u(n-2)
\\
&=\sum_{n=0}^{\infty}\brak{-\frac{1}{2}}^{n} + 
\sum_{n=2}^{\infty}\brak{-\frac{1}{2}}^{n-2}
\end{align} 
\\  From Infinite GP
\begin{align}
&=\frac{2}{3} + \frac{2}{3}\\&=\frac{4}{3}<\infty
\end{align}
Thus System is stable from \eqref{eq:stable}
\bigskip
%
\item Verify the above result using a python code.
\solution
\begin{lstlisting}
https://github.com/kurugodukarthik11/EE3900/blob/main/Assignments/Assignment_1/codes/5_6.py	
\end{lstlisting} 
\item 
Compute and sketch $h(n)$ using 
\begin{equation}
\label{eq:iir_filter_h}
h(n) + \frac{1}{2}h(n-1) = \delta(n) + \delta(n-2), 
\end{equation}
%
This is the definition of $h(n)$.
\\
\solution The following code plots Fig. \ref{fig:hndef}. Note that this is the same as Fig. 
\ref{fig:hn}. 
%
\begin{lstlisting}
wget https://github.com/kurugodukarthik11/EE3900/blob/main/Assignments/Assignment_1/codes/hndef.py	
\end{lstlisting}
\begin{figure}[!ht]
\centering
\includegraphics[width=\columnwidth]{./figs/hndef}
\caption{$h(n)$ from the definition}
\label{fig:hndef}
\end{figure}
%
\item Compute 
%
\begin{equation}
\label{eq:convolution}
y(n) = x(n)*h(n) = \sum_{n=-\infty}^{\infty}x(k)h(n-k)
\end{equation}
%
Comment. The operation in \eqref{eq:convolution} is known as
{\em convolution}.
%
\\
\solution The following code plots Fig. \ref{fig:ynconv}. Note that this is the same as 
$y(n)$ in  Fig. 
\ref{fig:xnyn}. 
%
\begin{lstlisting}
wget https://raw.githubusercontent.com/gadepall/EE1310/master/filter/codes/ynconv.py
\end{lstlisting}
\begin{figure}[!ht]
\centering
\includegraphics[width=\columnwidth]{./figs/ynconv}
\caption{$y(n)$ from the definition of convolution}
\label{fig:ynconv}
\end{figure}
\item Express the above convolution using a Teoplitz matrix.
\\
\solution 
\\The Toeplitz matrices for convolution are,
\begin{align}
 \mtx{y} &= \mtx{x} \circledast \mtx{h}\\
 \mtx{y} &= 
 \begin{pmatrix}
  h_1 & 0 & . & . & . & 0 \\
  h_2 & h_1 & . & . & . & 0 \\
  h_3 & h_2 & h_1 & . & . & 0 \\
  . & . & . & . & . & . \\
  1 & . & . & h_3 & h_2 & h_1 \\
  2 & . & . & . & h_2 & h_1 \\
  3 & . & . & . & 0 & h_1
 \end{pmatrix}
 \begin{pmatrix}
  x_1 \\ x_2 \\ \vdots \\ x_n
 \end{pmatrix}
\end{align}
\item Show that
\begin{equation}
y(n) =  \sum_{n=-\infty}^{\infty}x(n-k)h(k)
\end{equation}
\\
\solution
\solution
From eqn \eqref{eq:convolution}
\begin{align}
    y(n) = x(n)*h(n) = \sum_{n=-\infty}^{\infty}x(k)h(n-k)
\end{align}
Replace 'k' with 'n-k'
\begin{align}
y(n) = x(n)*h(n) 
\\
y(n)= \sum_{n=-\infty}^{\infty}x(n-k)h(n-(n-k))
\\
y(n)=\sum_{n=-\infty}^{\infty}x(n-k)h(k)
\end{align}
Hence Proved
\end{enumerate}
%
\section{DFT and FFT}
\begin{enumerate}[label=\thesection.\arabic*]
\item
Compute
\begin{equation}
X(k) \define \sum _{n=0}^{N-1}x(n) e^{-\j2\pi kn/N}, \quad k = 0,1,\dots, N-1
	\end{equation}
and $H(k)$ using $h(n)$.\\
\solution
\begin{lstlisting}
wget https://github.com/kurugodukarthik11/EE3900/blob/main/Assignments/Assignment_1/codes/6_1.py
\end{lstlisting}
\item Compute 
\begin{equation}
Y(k) = X(k)H(k)
\end{equation}
\\
\solution
\bigskip
\begin{lstlisting}
wget https://github.com/kurugodukarthik11/EE3900/blob/main/Assignments/Assignment_1/codes/6_2.py
\end{lstlisting}
\item Compute
\begin{equation}
 y\brak{n}={\frac {1}{N}}\sum _{k=0}^{N-1}Y\brak{k}\cdot e^{\j 2\pi kn/N},\quad n = 0,1,\dots, N-1
\end{equation}
\\
\\
\solution The following code plots Fig. \ref{fig:ynconv}. Note that this is the same as 
$y(n)$ in  Fig. 
\ref{fig:xnyn}. 
%
\begin{lstlisting}
wget https://raw.githubusercontent.com/gadepall/EE1310/master/filter/codes/yndft.py
\end{lstlisting}
\begin{figure}[!ht]
\centering
\includegraphics[width=\columnwidth]{./figs/yndft}
\caption{$y(n)$ from the DFT}
\label{fig:yndft}
\end{figure}
\item Repeat the previous exercise by computing $X(k), H(k)$ and $y(n)$ through FFT and 
IFFT.
\\
\solution
\begin{lstlisting}
wget https://github.com/kurugodukarthik11/EE3900/blob/main/Assignments/Assignment_1/codes/6_4.py
\end{lstlisting}
\end{enumerate}
\section{FFT}
% \subsection{Definitions}
\begin{enumerate}[label=\arabic*.,ref=\thesection.\theenumi]
\numberwithin{equation}{section}
    \item The DFT of $x(n)$ is given by
    \begin{align}
        X(k) \triangleq \sum_{n=0}^{N-1} x(n) e^{-j 2 \pi k n / N}, \quad k=0,1, \ldots, N-1
    \end{align}
\item Let 
	\begin{align}
W_{N} = e^{-j2\pi/N} 
	\end{align}
		Then the $N$-point {\em DFT matrix} is defined as 
	\begin{align}
		\vec{F}_{N} = \sbrak{W_{N}^{mn}}, \quad 0 \le m,n \le N-1 
	\end{align}
	where $W_{N}^{mn}$ are the elements of $\vec{F}_{N}$.
\item Let 
	\begin{align}
		\vec{I}_4 = \myvec{\vec{e}_4^{1} &\vec{e}_4^{2} &\vec{e}_4^{3} &\vec{e}_4^{4} }
	\end{align}
		be the $4\times 4$ identity matrix.  Then the 4 point {\em DFT permutation matrix} is defined as 
	\begin{align}
		\vec{P}_4 = \myvec{\vec{e}_4^{1} &\vec{e}_4^{3} &\vec{e}_4^{2} &\vec{e}_4^{4} }
	\end{align}
\item The 4 point {\em DFT diagonal matrix} is defined as 
	\begin{align}
		\vec{D}_4 = diag\myvec{W_{8}^{0} & W_{8}^{1} & W_{8}^{2} & W_{8}^{3}}
	\end{align}
\item Show that 
\begin{equation}
    W_{N}^{2}=W_{N/2}
    \label{eq:n-2}
\end{equation}
\solution We write
\begin{align}
	W_N^2 = \brak{e^{-\frac{\j2\pi}{N}}}^2 = e^{-\frac{\j2\pi}{N/2}} = W_{N/2}
\end{align}
%    \item Find $\vec{P}_6$.
%    \item Find $\vec{D}_3$.
    \item Show that 
\begin{equation}	
	\vec{F}_{4}=
\begin{bmatrix}
	\vec{I}_{2} & \vec{D}_{2} \\
\vec{I}_{2} & -\vec{D}_{2}
\end{bmatrix}
\begin{bmatrix}
\vec{F}_{2} & 0 \\
0 & \vec{F}_{2}
\end{bmatrix}
\vec{P}_{4}
\label{eq:fft-recurrence}
\end{equation}
\solution Observe that for $n \in \mathbb{N}$, $W_4^{4n} = 1$ and $W_4^{4n + 2} = -1$. Using \eqref{eq:n-2},
\begin{align}
	\vec{D}_2\vec{F}_2 &= \mybvec{\w{4}{0} & 0 \\ 0 & \w{4}{1}}\mybvec{\w{2}{0} & \w{2}{0} \\ \w{2}{0} & \w{2}{1}} \\
					   &= \mybvec{\w{4}{0} & 0 \\ 0 & \w{4}{1}}\mybvec{\w{4}{0} & \w{4}{0} \\ \w{4}{0} & \w{4}{2}} \\
					   &= \mybvec{\w{4}{0} & \w{4}{0} \\ \w{4}{1} & \w{4}{3}} \label{eq:fft-df1} \\
	\implies -\vec{D}_2\vec{F}_2 &= \mybvec{\w{4}{2} & \w{4}{6} \\ \w{4}{3} & \w{4}{9}} \label{eq:fft-df2}
\end{align}
and
\begin{align}
	\vec{F}_2 &= \myvec{\w{2}{0} & \w{2}{0} \\ \w{2}{0} & \w{2}{1}} \\
			  &= \myvec{\w{4}{0} & \w{4}{0} \\ \w{4}{0} & \w{4}{2}}
\end{align}
Hence,
\begin{align}
	\vec{W}_4 &= \myvec{\w{4}{0} & \w{4}{0} & \w{4}{0} & \w{4}{0} \\
		\w{4}{0} & \w{4}{2} & \w{4}{1} & \w{4}{3} \\
		\w{4}{0} & \w{4}{4} & \w{4}{2} & \w{4}{6} \\
		\w{4}{0} & \w{4}{6} & \w{4}{3} & \w{4}{9} 
	} \label{eq:fft-permutation} \\
	&= \mybvec{\vec{I}_2\vec{F}_2 & \vec{D}_2{F}_2 \\ \vec{I}_2\vec{F}_2 & -\vec{D}_2{F}_2} \\
	&= \mybvec{\vec{I}_2 & \vec{D}_2 \\ \vec{I}_2 & \vec{D}_2}\mybvec{\vec{F}_2 & 0 \\ 0 & \vec{F}_2}
	\label{eq:ifd}
\end{align}
Multiplying \eqref{eq:ifd} by $\vec{P}_4$ on both sides, and noting that $\vec{W}_4\vec{P}_4 = \vec{F}_4$ gives us \eqref{eq:fft-recurrence}.
\item Show that 
\begin{equation}
\vec{F}_{N}=
\begin{bmatrix}
\vec{I}_{N/2} & \vec{D}_{N/2} \\
\vec{I}_{N/2} & -\vec{D}_{N/2}
\end{bmatrix}
\begin{bmatrix}
\vec{F}_{N/2} & 0 \\
0 & \vec{F}_{N/2}
\end{bmatrix}
\vec{P}_{N}
\end{equation}
\solution Observe that for even $N$ and letting $\vec{f}_N^i$ denote the $i^{\text{th}}$ column of $\vec{F}_N$, from \eqref{eq:fft-df1} and \eqref{eq:fft-df2},
\begin{align}
	\myvec{\vec{D}_{N/2}\vec{F}_{N/2} \\ -\vec{D}_{N/2}\vec{F}_{N/2}} = \myvec{\vec{f}_N^{2} & \vec{f}_N^{4} & \ldots & \vec{f}_N^{N}}
\end{align}
and
\begin{align}
	\myvec{\vec{I}_{N/2}\vec{F}_{N/2} \\ \vec{I}_{N/2}\vec{F}_{N/2}} = \myvec{\vec{f}_N^{1} & \vec{f}_N^{3} & \ldots & \vec{f}_N^{N - 1}}
\end{align}
Thus,
\begin{align}
	&\mybvec{\vec{I}_2\vec{F}_2 & \vec{D}_2\vec{F}_2 \\ \vec{I}_2\vec{F}_2 & -\vec{D}_2\vec{F}_2} = \mybvec{\vec{I}_{N/2} & \vec{D}_{N/2} \\ \vec{I}_{N/2} & -\vec{D}_{N/2}}\mybvec{\vec{F}_{N/2} & 0 \\ 0 & \vec{F}_{N/2}} \nonumber \\
	&= \myvec{\vec{f}_N^{1} & \ldots & \vec{f}_N^{N - 1} & \vec{f}_N^{2} & \ldots & \vec{f}_N^{N}}
\end{align}
and so,
\begin{align}
	&\mybvec{\vec{I}_{N/2} & \vec{D}_{N/2} \\ \vec{I}_{N/2} & -\vec{D}_{N/2}}\mybvec{\vec{F}_{N/2} & 0 \\ 0 & \vec{F}_{N/2}}\vec{P}_{N} \nonumber \\
	&= \myvec{\vec{f}_N^{1} & \vec{f}_N^{2} & \ldots & \vec{f}_N^{N}} = \vec{F}_N
\end{align}
\item Find 
    \begin{align}
	     \vec{P}_4 \vec{x}
    \end{align}
\solution We have,
\begin{align}
	\vec{P}_4\vec{x} = \myvec{\vec{e}_4^1 & \vec{e}_4^3 & \vec{e}_4^2 & \vec{e}_4^4}\myvec{x(0)\\x(1)\\x(2)\\x(3)} = \myvec{x(0)\\x(2)\\x(1)\\x(3)}
	\label{eq:x-permute}
\end{align}
%\item Find 
 %   \begin{align}
	%     \vec{P}_4 \vec{x}
    %\end{align}
\item Show that 
    \begin{align}
	    \vec{X} = \vec{F}_N \vec{x}
	    \label{eq:dft-mat-def}
    \end{align}
		where $\vec{x}, \vec{X}$ are the vector representations of $x(n), X(k)$ respectively.
\solution Writing the terms of $X$, 
\begin{align}
	X(0) &= x(0) + x(1) + \ldots + x(N - 1) \\
	X(1) &= x(0) + x(1)e^{-\frac{\j2\pi}{N}} + \ldots + \nonumber \\
		 &+ x(N - 1)e^{-\frac{\j2(N - 1)\pi}{N}} \\
		 &\vdots \nonumber \\
	X(N - 1) &= x(0) + x(1)e^{-\frac{\j2(N - 1)\pi}{N}} + \ldots + \nonumber \\
			 &+ x(N - 1)e^{-\frac{\j2(N - 1)(N - 1)\pi}{N}}	
\end{align}
Clearly, the term in the $m^{\text{th}}$ row and $n^{\text{th}}$ column is given by ($0 \leq m \leq N - 1$ and $0 \leq n \leq N - 1$) 
\begin{align}
	T_{mn} = x(n)e^{-\frac{\j2mn\pi}{N}} 
\end{align}
and so, we can represent each of these terms as a matrix product
\begin{align}
	\vec{X} = \vec{F}_N\vec{x}
\end{align}
where $\vec{F}_N = \sbrak{e^{-\frac{-\j2mn\pi}{N}}}_{mn}$ for $0 \leq m \leq N - 1$ and $0 \leq n \leq N - 1$. 
\item Derive the following Step-by-step visualisation  of
8-point FFTs into 4-point FFTs and so on
\begin{equation}
\begin{bmatrix}
X(0) \\ 
X(1) \\ 
X(2) \\ 
X(3)
\end{bmatrix}
=
\begin{bmatrix}
X_{1}(0) \\ 
X_{1}(1)\\ 
X_{1}(2)\\
X_{1}(3)\\
\end{bmatrix}
+
\begin{bmatrix}
W^{0}_{8} & 0 & 0 & 0\\
0 & W^{1}_{8} & 0 & 0\\
0 & 0 & W^{2}_{8} & 0\\
0 & 0 & 0 & W^{3}_{8}
\end{bmatrix}
\begin{bmatrix}
X_{2}(0) \\ 
X_{2}(1) \\ 
X_{2}(2) \\
X_{2}(3)
\end{bmatrix}
\label{eq:8-low}
\end{equation}
\begin{equation}
\begin{bmatrix}
X(4) \\ 
X(5) \\ 
X(6) \\ 
X(7)
\end{bmatrix}
=
\begin{bmatrix}
X_{1}(0) \\ 
X_{1}(1)\\ 
X_{1}(2)\\
X_{1}(3)\\
\end{bmatrix}
-
\begin{bmatrix}
W^{0}_{8} & 0 & 0 & 0\\
0 & W^{1}_{8} & 0 & 0\\
0 & 0 & W^{2}_{8} & 0\\
0 & 0 & 0 & W^{3}_{8}
\end{bmatrix}
\begin{bmatrix}
X_{2}(0) \\ 
X_{2}(1) \\ 
X_{2}(2) \\
X_{2}(3)
\end{bmatrix}
\label{eq:8-high}
\end{equation}
4-point FFTs into 2-point FFTs
\begin{equation}
\begin{bmatrix}
X_{1}(0) \\ 
X_{1}(1)\\ 
\end{bmatrix}
=
\begin{bmatrix}
X_{3}(0) \\ 
X_{3}(1)\\ 
\end{bmatrix}
+
\begin{bmatrix}
W^{0}_{4} & 0\\
0 & W^{1}_{4}
\end{bmatrix}
\begin{bmatrix}
X_{4}(0) \\ 
X_{4}(1) \\ 
\end{bmatrix}
\end{equation}
\begin{equation}
\begin{bmatrix}
X_{1}(2) \\ 
X_{1}(3)\\ 
\end{bmatrix}
=
\begin{bmatrix}
X_{3}(0) \\ 
X_{3}(1)\\ 
\end{bmatrix}
-
\begin{bmatrix}
W^{0}_{4} & 0\\
0 & W^{1}_{4}
\end{bmatrix}
\begin{bmatrix}
X_{4}(0) \\ 
X_{4}(1) \\ 
\end{bmatrix}
\end{equation}
\begin{equation}
\begin{bmatrix}
X_{2}(0) \\ 
X_{2}(1)\\ 
\end{bmatrix}
=
\begin{bmatrix}
X_{5}(0) \\ 
X_{5}(1)\\ 
\end{bmatrix}
+
\begin{bmatrix}
W^{0}_{4} & 0\\
0 & W^{1}_{4}
\end{bmatrix}
\begin{bmatrix}
X_{6}(0) \\ 
X_{6}(1) \\ 
\end{bmatrix}
\end{equation}
\begin{equation}
\begin{bmatrix}
X_{2}(2) \\ 
X_{2}(3)\\ 
\end{bmatrix}
=
\begin{bmatrix}
X_{5}(0) \\ 
X_{5}(1)\\ 
\end{bmatrix}
-
\begin{bmatrix}
W^{0}_{4} & 0\\
0 & W^{1}_{4}
\end{bmatrix}
\begin{bmatrix}
X_{6}(0) \\ 
X_{6}(1) \\ 
\end{bmatrix}
\end{equation}
\begin{equation}
P_{8}
\begin{bmatrix}
x(0) \\ 
x(1) \\ 
x(2) \\ 
x(3) \\ 
x(4) \\ 
x(5) \\
x(6) \\
x(7)
\end{bmatrix}
 = 
\begin{bmatrix}
x(0) \\ 
x(2) \\ 
x(4) \\ 
x(6) \\
x(1) \\ 
x(3) \\ 
x(5) \\
x(7)
\end{bmatrix}
\end{equation}
\begin{equation}
P_{4}
\begin{bmatrix}
x(0) \\ 
x(2) \\ 
x(4) \\ 
x(6) \\
\end{bmatrix}
 = 
\begin{bmatrix}
x(0) \\ 
x(4) \\ 
x(2) \\
x(6)
\end{bmatrix}
\end{equation}
\begin{equation}
P_{4}
\begin{bmatrix}
x(1) \\ 
x(3) \\ 
x(5) \\
x(7)
\end{bmatrix}
 = 
\begin{bmatrix}
x(1) \\ 
x(5) \\ 
x(3) \\ 
x(7) \\
\end{bmatrix}
\end{equation}
Therefore,
\begin{equation}
\begin{bmatrix}
X_{3}(0) \\ 
X_{3}(1)\\ 
\end{bmatrix}
= F_{2}
\begin{bmatrix}
x(0) \\ 
x(4) \\ 
\end{bmatrix}
\end{equation}
\begin{equation}
\begin{bmatrix}
X_{4}(0) \\ 
X_{4}(1)\\ 
\end{bmatrix}
= F_{2}
\begin{bmatrix}
x(2) \\ 
x(6) \\ 
\end{bmatrix}
\end{equation}
\begin{equation}
\begin{bmatrix}
X_{5}(0) \\ 
X_{5}(1)\\ 
\end{bmatrix}
= F_{2}
\begin{bmatrix}
x(1) \\ 
x(5) \\ 
\end{bmatrix}
\end{equation}
\begin{equation}
\begin{bmatrix}
X_{6}(0) \\ 
X_{6}(1)\\ 
\end{bmatrix}
= F_{2}
\begin{bmatrix}
x(3) \\ 
x(7) \\ 
\end{bmatrix}
\end{equation}
\solution We write out the values of performing an 8-point FFT on $\vec{x}$ as follows.
\begin{align}
	X(k) &= \sum_{n = 0}^{7}x(n)e^{-\frac{\j2kn\pi}{8}} \\
		 &= \sum_{n = 0}^{3}\brak{x(2n)e^{-\frac{\j2kn\pi}{4}} + e^{-\frac{\j2k\pi}{8}}x(2n + 1)e^{-\frac{\j2kn\pi}{4}}} \\
		 &= X_1(k) + e^{-\frac{\j2k\pi}{4}}X_2(k) 
\end{align}
where $\vec{X}_1$ is the 4-point FFT of the even-numbered terms and $\vec{X}_2$ is the 4-point FFT of the odd numbered terms. Noticing that for $k \geq 4$,
\begin{align}
	X_1(k) &= X_1(k - 4) \\
	e^{-\frac{\j2k\pi}{8}} &= -e^{-\frac{\j2(k - 4)\pi}{8}}
\end{align}
we can now write out $X(k)$ in matrix form as in \eqref{eq:8-low} and \eqref{eq:8-high}. We also need to solve the two 4-point FFT terms so formed.
\begin{align}
	X_1(k) &= \sum_{n = 0}^{3}x_1(n)e^{-\frac{\j2kn\pi}{8}} \\
		 &= \sum_{n = 0}^{1}\brak{x_1(2n)e^{-\frac{\j2kn\pi}{4}} + e^{-\frac{\j2k\pi}{8}}x_2(2n + 1)e^{-\frac{\j2kn\pi}{4}}} \\
		 &= X_3(k) + e^{-\frac{\j2k\pi}{4}}X_4(k) 
\end{align}
using $x_1(n) = x(2n)$ and $x_2(n) = x(2n + 1)$. Thus we can write the 2-point FFTs
\begin{align}
\begin{bmatrix}
X_{3}(0) \\ 
X_{3}(1)\\ 
\end{bmatrix}
= F_{2}
\begin{bmatrix}
x(0) \\ 
x(4) \\ 
\end{bmatrix} \\
\begin{bmatrix}
X_{4}(0) \\ 
X_{4}(1)\\ 
\end{bmatrix}
= F_{2}
\begin{bmatrix}
x(2) \\ 
x(6) \\ 
\end{bmatrix}
\end{align}
Using a similar idea for the terms $X_2$, 
\begin{align}
\begin{bmatrix}
X_{5}(0) \\ 
X_{5}(1)\\ 
\end{bmatrix}
= F_{2}
\begin{bmatrix}
x(1) \\ 
x(5) \\ 
\end{bmatrix} \\
\begin{bmatrix}
X_{6}(0) \\ 
X_{6}(1)\\ 
\end{bmatrix}
= F_{2}
\begin{bmatrix}
x(3) \\ 
x(7) \\ 
\end{bmatrix}
\end{align}
But observe that from \eqref{eq:x-permute},
\begin{align}
	\vec{P}_8\vec{x} &= \myvec{\vec{x}_1\\\vec{x}_2} \\
	\vec{P}_4\vec{x}_1 &= \myvec{\vec{x}_3\\\vec{x}_4} \\ 
	\vec{P}_4\vec{x}_2 &= \myvec{\vec{x}_5\\\vec{x}_6}
\end{align}
where we define $x_3(k) = x(4k)$, $x_4(k) = x(4k + 2)$, $x_5(k) = x(4k + 1)$, and $x_6(k) = x(4k + 3)$ for $k = 0, 1$.
\item For 
    \begin{align}
	    \vec{x} = \myvec{1\\2\\3\\4\\2\\1}
        \label{eq:equation1}
    \end{align}
    compte the DFT  
		using 
	    \eqref{eq:dft-mat-def}
\solution Download the Python code from 
\begin{lstlisting}
$ wget https://github.com/kurugodukarthik11/EE3900/blob/main/Assignments/Assignment_1/codes/7_11.py
\end{lstlisting}
\item Repeat the above exercise using the FFT
	    after zero padding $\vec{x}$.
%	    \eqref{eq:fft-mat-def}
\item Write a C program to compute the 8-point FFT.
\begin{lstlisting}
$ wget https://github.com/kurugodukarthik11/EE3900/blob/main/Assignments/Assignment_1/codes/7_13.c
\end{lstlisting}
\begin{figure}[!ht]
\centering
\includegraphics[width=\columnwidth]{/home/poseidon/SEM V/EE3900/Assignment_1/filter/figs/7_13.pdf}
\caption{Complexity Analysis}
\end{figure} 
\end{enumerate}
 
 
\section{Exercises}
Answer the following questions by looking at the python code in Problem \ref{prob:output}.
\begin{enumerate}[label=\thesection.\arabic*]
\item
The command
\begin{lstlisting}
	output_signal = signal.lfilter(b, a, input_signal)
	\end{lstlisting}
in Problem \ref{prob:output} is executed through the following difference equation
\begin{equation}
\label{eq:iir_filter_gen}
 \sum _{m=0}^{M}a\brak{m}y\brak{n-m}=\sum _{k=0}^{N}b\brak{k}x\brak{n-k}
\end{equation}
%
where the input signal is $x(n)$ and the output signal is $y(n)$ with initial values all 0. Replace
\textbf{signal.filtfilt} with your own routine and verify.\\
\solution
On taking the $Z$-transform on both sides of the difference equation
	\begin{align}
		\sum _{m=0}^{M}a\brak{m} z^{-m} Y(z) &= \sum _{k=0}^{N}b\brak{k} z^{-k} X(z) \\
		\implies H(z) = \frac{Y(z)}{X(z)} &= \frac{\sum _{k=0}^{N}b\brak{k} z^{-k}}{\sum _{m=0}^{M}a\brak{m} z^{-m	}}
	\end{align}
	For obtaining the discrete Fourier transform, put $z = \j \frac{2\pi i}{I}$ where $I$ is the length of the input signal and $i = 0, 1, \ldots, I-1$
Run the following code 
\begin{lstlisting}
wget https://github.com/kurugodukarthik11/EE3900/blob/main/Assignments/Assignment_1/codes/8_1.py
\end{lstlisting}
\item Repeat all the exercises in the previous sections for the above $a$ and $b$.\\
\solution
The polynomial coefficients obtained are
	\begin{align}
		\vec{a} = \myvec{1.000 \\ -2.519 \\ 2.561 \\ -1.206 \\ 0.220} \qquad
		\vec{b} = \myvec{0.003 \\ 0.014 \\ 0.021 \\ 0.014 \\ 0.003}
	\end{align}
	
	The difference equation is then given by
	\begin{multline}
		\vec{a}^\top \vec{y} = \vec{b}^\top \vec{x} \\
		y(n)-(2.519)y(n-1)+(2.561)y(n-2)\\-(1.206)y(n-3)+(0.220)y(n-4)=
		(0.003)x(n)\\+(0.014)x(n-1)+(0.021)x(n-2)+(0.014)x(n-3)\\+(0.003)x(n-4)
	\end{multline}
	
	where
	\begin{align}
		\vec{y} = \myvec{y(n) \\ y(n-1) \\ y(n-2) \\ y(n-3) \\ y(n-4)} \qquad
		\vec{x} = \myvec{x(n) \\ x(n-1) \\ x(n-2) \\ x(n-3) \\ x(n-4)}
	\end{align}
Run the following code 
\begin{lstlisting}
wget https://github.com/kurugodukarthik11/EE3900/blob/main/Assignments/Assignment_1/codes/8_2a.py
\end{lstlisting}
\begin{figure}[!ht]
\centering
\includegraphics[width=\columnwidth]{/home/poseidon/SEM V/EE3900/Assignment_1/filter/figs/8_2i.pdf}
\caption{$|H(e^{j\omega})|$}
\end{figure}
Run the following code 
\begin{lstlisting}
wget https://github.com/kurugodukarthik11/EE3900/blob/main/Assignments/Assignment_1/codes/8_2b.py
\end{lstlisting}
\begin{figure}[!ht]
\centering
\includegraphics[width=\columnwidth]{/home/poseidon/SEM V/EE3900/Assignment_1/filter/figs/8_1.png}
\caption{h(n)}
\end{figure}
\item What is the sampling frequency of the input signal?
\\
\solution
Sampling frequency(fs)=44.1kHZ.
\item
What is type, order and  cutoff-frequency of the above butterworth filter
\\
\solution
The given butterworth filter is low pass with order=4 and cutoff-frequency=4kHz.
%
\item
Modifying the code with different input parameters and to get the best possible output.\\
%
\solution 
Order: 10\\
Cutoff frequency: 3000 Hz
\end{enumerate}
\end{document}
